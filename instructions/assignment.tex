% causal broadcast chat
% extension tasks?
%
% whether it works
% whether it's fault tolerant
% not client server
% quality of code
% quality of report
%


\documentclass[a4paper]{article}

% Use utf-8 encoding for foreign characters
\usepackage[utf8]{inputenc}

% Setup for fullpage use
% \usepackage{fullpage}

% Uncomment some of the following if you use the features
%
% Running Headers and footers
%\usepackage{fancyhdr}

% Multipart figures
%\usepackage{subfigure}

% More symbols
% \usepackage{amsmath}
% \usepackage{amssymb}
%\usepackage{latexsym}
\newcommand{\union}{\cup}

% Surround parts of graphics with box
% \usepackage{boxedminipage}

% Package for including code in the document
\usepackage{listings}

\usepackage[ruled,vlined]{algorithm2e}

\usepackage{hyperref}


% This is now the recommended way for checking for PDFLaTeX:
\usepackage{ifpdf}

%\newif\ifpdf
%\ifx\pdfoutput\undefined
%\pdffalse % we are not running PDFLaTeX
%\else
%\pdfoutput=1 % we are running PDFLaTeX
%\pdftrue
%\fi

\ifpdf
\usepackage[pdftex]{graphicx}
\else
\usepackage{graphicx}
\fi
\title{LSINF2345 Assignment 2013: Dancing robots}
\author{Nicholas Rutherford}

\date{April 2013}


\begin{document}
\lstset{language=erlang}

\ifpdf
\DeclareGraphicsExtensions{.pdf, .jpg, .tif}
\else
\DeclareGraphicsExtensions{.eps, .jpg}
\fi

\maketitle


\section{Introduction}

Shamshung's robotics team are deploying a new publicity campaign this year:
georeplicated dancing robots. The robots will dance in 4 different venues
simultaneously, connected by video, making up their own dance routine live.
They have to be careful that nothing goes wrong, as a major rival,
Babble, will be watching carefully for mistakes to discuss over lunch with
their customers.

Unfortunately for Shamsung, their robotic dancers are rather temperamental,
and prone to breakdown before a show, or spectacular (often explosive) failure
during a performance. This is unfortunate because, while naively assuming
their robots would live forever, the team's software engineers decided one
robot would lead, deciding on a dance routine and telling the other robots
which steps to perform. This is not predetermined: the robots decide the next
step just before taking it.

In order to keep the show moving we need to substitute this
central-point-of-failure with a suitable alternative. In this assignment you
will design two (or more) solutions which keep the robots dancing using
distributed algorithms from the course. You should explain your approach and
provide a prototype implementation using the provided Erlang stack and
dancing robot simulator.

Management's specification for the dancers:

  - the robots will start from the same pose
  - the robots will perform the same dance routine: the same steps in the same order,
    as if it had been written down before the show and given to them
  - no robot can miss a step: if one robot takes it they all take its
  - all robots must finish in the same pose and the same place on the stage

  - (after the show the producer wants a print-out of the dance routine (to sell at auction))
  - (no two robots can be out-of-sync by more than 10 seconds.)
  - (create an application layer which selects a robot from the group to 
      display on the local TV screen. It should announce a robot every 10 seconds, 
      and should eventually stop displaying crashed robots).




\subsection{Installing Erlang and WX} % (fold)
\label{sub:installing_erlang_and_wx}

Be sure to install (or build) Erlang with wxWidgets for the graphical
component of the assignment.

Mac users (10.8) can install 32-bit Erlang from
\url{https://www.erlang-solutions.com/downloads/download-erlang-otp}.



% subsection installing_erlang_and_wx (end)



\subsection{Where to find help} % (fold)
\label{sub:where_to_find_help}

In order to find documentation for Erlang functions, try Google, the reference
manual\footnote{\url{http://www.erlang.org/doc/apps/stdlib/index.html}}, and
possibly \url{http://erldocs.com}.
To better understand the algorithms, refer to the course slides or textbook
\cite{cachin2011}.

You can email me questions about your design and problems with Erlang. Please
also arrange by email if you want to meet in person.

It's ok to talk to the other students about how to solve the problems, but
don't copy their code.

% subsection where_to_find_help (end)








\section{Optimising the perfect-link abstraction} % (fold)
\label{sub:optimising_pl}

The \emph{pl}, \emph{sl}, \emph{fll} messaging stack is theoretically correct, but expensive.
In the following you'll optimise your stack to send fewer
messages and consider how it changes the abstraction's properties.

\subsection{Weakening stubborn delivery by halting retransmission} % (fold)
\label{ssub:weakening_stubborn_delivery_by_halting_retransmission}

Does the ``stubborn delivery'' property make sense in practice? Can you think
of a time when it's safe to stop retransmitting the same message? How would
you implement it?

% subsubsection weakening_stubborn_delivery_by_halting_retransmission (end)

\subsection{Garbage collecting terminated algorithm retransmissions} % (fold)
\label{ssub:halting_retransmission_on_algorithm_termination}

Having found a way to halt the retransmission of individual messages, can you
think of a way to modify this to work on sets of messages? Once an algorithm
using the links has terminated execution there is no need to continue
retransmitting its messages. For now don't worry about algorithm termination,
simply demonstrate that you can group messages by an integer identifier (or
similar) and have them all stop being retransmitted. That is to say, you
should modify ``If a correct process p sends a message m once to a correct
process q, then q delivers m an infinite number of times'' to ``If a correct
process p sends a message m once to a correct process q as part of an
algorithm a, then q delivers m one or more times before a terminates, and
infinitely often if a does not terminate''.

% subsubsection halting_retransmission_on_algorithm_termination (end)

\subsection{Other ideas} % (fold)
\label{ssub:other_ideas}

How else might you optimise \emph{pl}, \emph{sl}, \emph{fll}? Check the performance clauses of
sections 2.4.3-4 in the course book, about \emph{sl} and \emph{pl}, for ideas.

Consider the consequences of node failures and the various failure models
with the algorithms and your changes.

% subsubsection other_ideas (end)

% subsection optimising_pl (end)




\end{document}

