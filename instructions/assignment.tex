% causal broadcast chat
% extension tasks?
%
% whether it works
% whether it's fault tolerant
% not client server
% quality of code
% quality of report
%


\documentclass[a4paper]{article}

% Use utf-8 encoding for foreign characters
\usepackage[utf8]{inputenc}

% Setup for fullpage use
\usepackage{a4wide}

% Uncomment some of the following if you use the features
%
% Running Headers and footers
%\usepackage{fancyhdr}

% Multipart figures
%\usepackage{subfigure}

% More symbols
\usepackage{amsmath}
\usepackage{amssymb}
%\usepackage{latexsym}
\newcommand{\union}{\cup}

% Surround parts of graphics with box
% \usepackage{boxedminipage}

% Package for including code in the document
\usepackage{listings}
\lstset{language=erlang}

% prevent latex destroying ' and " which gives copy-paste code errors
\usepackage{upquote}

\usepackage[ruled,vlined]{algorithm2e}

\usepackage{hyperref}


% This is now the recommended way for checking for PDFLaTeX:
\usepackage{ifpdf}

%\newif\ifpdf
%\ifx\pdfoutput\undefined
%\pdffalse % we are not running PDFLaTeX
%\else
%\pdfoutput=1 % we are running PDFLaTeX
%\pdftrue
%\fi

\ifpdf
\usepackage[pdftex]{graphicx}
\else
\usepackage{graphicx}
\fi
\title{LSINF2345 Assignment 2013: Dancing robots}
\author{Nicholas Rutherford, Université catholique de Louvain}

\date{April 2013}


\begin{document}

\ifpdf
\DeclareGraphicsExtensions{.pdf, .jpg, .tif}
\else
\DeclareGraphicsExtensions{.eps, .jpg}
\fi

\maketitle


\section*{Introduction}

Shamshung is a global electronics producer, priding itself on state-of-the-art
technology. One of their directors recently discovered a video of dancing
robots\footnote{\url{https://www.youtube.com/watch?v=4t1NWH6G1f0}}, and has
decided to one-up their competitor. They've given the following specification:

\begin{itemize}

  \item[\emph{Entertaining}] The robots cannot stand still. They must eventually dance, then dance until the halting condition is met.

  \item[\emph{Halting}] The dance should stop at 100 steps.

  \item[\emph{Coordination}] The robots start from the same position, and dance the same sequence: the same steps in the same order.

  \item[\emph{Completeness}] No correct robot can miss a step.

  \item[\emph{Finishing position}] All correct robots must finish dancing in the same
  pose and position.

  \item[\emph{Improvisation}] The sequence of dance steps is created by the robots
  \emph{during} each performance. It is not known beforehand.

  \item[\emph{Spontaneity}] The robots can plan no more than 5 steps in advance, so we can change the song.

\end{itemize}

Shamshung's robotics team rose to the challenge, and produced a number of
androids capable of dancing like The Village People, with the added
ability to shrink and expand (but not bang?) their heads along to the music.

A private screening for the director was a success, and led to a successful
public performance at a local technology expo. The robots were a big hit, and
as a reward for their hard work the robotics team were given a new challenge:
world tour! However, there is a twist. Shamshung's head of telecoms research
is a friend of the director in question, and convinced them that it is a great
marketing opportunity for their video-conferencing and mobile comms products
if they split the robots up and have them dancing together in different
countries, connected live by video-conferencing!

Naturally this led to a geo-replicated nightmare for the robot technicians,
who watched in horror as the robots began to violate all but one of their
specifications. Indeed, they were only able to keep their jobs by arguing that
the robots were performing an elaborate, individualistic, contemporary dance.
Some of the directors believed them. The others knew better, and hired you to
find out what went wrong and fix it before the next show.


% Shamshung's robotics team are deploying a new publicity campaign this year:
% georeplicated dancing robots. The robots will dance in 4 different venues
% simultaneously, connected by video, making up their own dance routine live.
% They have to be careful that nothing goes wrong, as a major rival,
% Babble, will be watching carefully for mistakes to discuss over lunch with
% their customers.
%
% Unfortunately for Shamsung, their robotic dancers turned out to be rather
% temperamental, and are prone to spectacular -- often explosive -- failure
% during performances. This is unfortunate because the team's software engineers
% decided one robot would lead, deciding on a dance routine and telling the
% other robots which steps to perform. This is not predetermined: the robots
% decide the next step just before taking it. They had assumed their robots
% would live forever.
%
% In order to keep the show moving we need to substitute this
% central-point-of-failure with a suitable alternative. In this assignment you
% will design two (or more) solutions which keep the robots dancing using
% distributed algorithms from the course. You should explain your approach and
% provide a prototype implementation using the provided Erlang stack and
% dancing robot simulator.




\section*{Practical details} % (fold)
\label{sec:practical_details}

To familiarise yourself with the programming framework and basic abstractions
be sure to complete the lab ``LSINF2345 Lab: Implementing the basic
abstractions''.

\subsection*{Running the software} % (fold)
\label{sub:running_the_software}

Note that you will need to implement some missing components (\emph{pl},
\emph{beb}) before the software will run.

\begin{verbatim}
  # Open a new terminal and cd to the project directory
  erl -sname 'n1@localhost' # for n1..nN with N dancers

  % In each erl shell:
  cd(ebin).
  stack:connect('n1@localhost').

  % Then, after all are connected, on each node
  stack:start_link().
  stack:add_component(dancing_robots).

  % once are stacks are launched
  dancing_robots:start_dance().
\end{verbatim}


% subsection running_the_software (end)

\subsection*{Deliverables} % (fold)
\label{sub:deliverables}

Turn in a zip file containing your \emph{src}, \emph{include} and \emph{test}
directories, and a PDF report. Word counts stated in the problems are soft
limits, but try not to be too far off. Optionally, you may include up to 2
pages worth of diagrams. A cover page is optional.

Your software will be tested with 32-bit Erlang R16B.

% subsection deliverables (end)

\subsection*{Marking} % (fold)
\label{sub:marking}

The assignment is out of a maximum of 20 points. The points for each section
are explained in their ``deliverables'' section. The main work is to complete
problems 0-3, which are worth a total of 15 points. 1 discretionary point is
available for a nicely presented report and implementation code. If you
are careful with your program and submit a nice report you can get 16/20
without attempting the extension work. Completing all of the extension tasks
makes it possible to get a very good score without having perfect answers for
every activity.

% subsection subsection_name (end)


\subsection*{Installing Erlang with wxWidgets} % (fold)
\label{sub:installing_erlang_and_wx}

Be sure to install (or build) Erlang with wxWidgets for the graphical
component of the assignment, otherwise the program will not compile and you
will get strange error messages when trying to run the stack.

Mac users (10.8) should use 32-bit Erlang from
\url{https://www.erlang-solutions.com/downloads/download-erlang-otp}. 64-bit
Erlang's wxWidget library does not work on recent Mac OSes. This
company also provides builds for other platforms.

If there are problems please let me know well in advance of the deadline.
That means install now and check wx works, even if you plan to procrastinate
(not advised).

% subsection installing_erlang_and_wx (end)


\subsection*{Where to find help} % (fold)
\label{sub:where_to_find_help}

In order to find documentation for Erlang functions, try Google, the reference
manual\footnote{\url{http://www.erlang.org/doc/apps/stdlib/index.html}}, and
possibly \url{http://erldocs.com}.
To better understand the algorithms, refer to the course slides or textbook
\cite{cachin2011}.
The assistant will announce their availability separately.
It's ok to talk to the other students about how to solve the problems, but
don't copy their code.

% subsection where_to_find_help (end)



% section practical_details (end)



\section*{Problem 0: Setting the stage, the uncoordinated robots} % (fold)
\label{sec:section_name}

We start by reproducing the naïve software, where no robot fails and no
special consideration has been given to broadcasting messages. That is, the
designer had not studied distributed computing.

Implement best-effort broadcast, \emph{beb}, and observe the behaviour of the
dancing robots in the GUI for a number of processes of your choice.

It is possible that you will already observe strange behaviour. Think about
the properties of \emph{beb}, and the way in which the robots are periodically
broadcasting what they think should be the next dance move.

\subsection*{Deliverables [1 point]} % (fold)
\label{sub:p0_deliverables}

First we will look at your broadcast and point-to-point link code, and
visually inspect the behaviour of your robots when reproducing the demo given
during the lab. The robots should dance in a (mostly) synchronised fashion
when losses and delays are turned off in \emph{fll}.

Optionally, you can write no more than 150 words about any strange behaviour
observed or problems you have spotted with this design.

% subsection deliverables (end)

% section section_name (end)



\section*{Problem 1: Lost and delayed messages, disorganised dancing} % (fold)
\label{sec:dealing_with_unreliable_links}

Having reproduced the team's design, let's investigate how it will behave on
a more realistic network. We will simulate the delay and loss of messages by
changing the ``reliable'' state variable of \emph{fll} from true to false.
This introduces arbitrary transmission delays and occasional lost messages.
Run the dancing robot  simulation again, and observe how its behaviour has
changed. Are the invariants of the product specification maintained or
violated?

You will soon realise that message ordering is important to maintain the state
of our robots (indeed, they are essentially a replicated-state-machine): the
dance steps must be totally ordered across the robots.

Keeping with the omission failure model, design two solutions to the total
ordering problem.
\begin{enumerate}

  \item A nominated leader decides the dance steps, broadcasts them with a
  point-to-point (but not global) ordering guarantee

  \item Allow all robots to suggest dance steps, as before, but totally order
  delivery and execution of their broadcasts

\end{enumerate}

\subsection*{Deliverable [5 points]} % (fold)
\label{sub:p1_deliverable}

Explain your solutions in your report. Aim for about one page.
In particular, explain how you have addressed the new
problems of dropped and delayed messages.

Implement both of them in modified (renamed) copies of the
\emph{dancing\_robots} module. Name them \emph{dancing\_leader} and
\emph{dancing\_bcast}. Optionally, in a further 150 or fewer words, describe any
interesting implementation details or challenges you faced.

Your software will be tested by checking the finishing position and dance
sequence of each robot in the group. You can access the steps and position
with the event \verb!{dancerobot, get_steps}!, which currently prints them
to terminal.

% subsection deliverable (end)

% section dealing_with_unreliable_links (end)










\section*{Problem 2: Failing nodes with a bounded (synchronous) network} % (fold)
\label{sec:problem_2_failing_nodes}

To make matters worse, it turns out that the robots do not travel well.
Whatever the problem is, it has made the dancers rather temperamental, and
prone to spectacular -- even explosive -- failure. The engineers are reluctant
to discuss this, so it's not clear which or how many robots will fail during
a performance. Despite this, the show must go on.

Modify one of your problem 1 solutions to survive failing robots. Assume
fail-stop (crash-stop) failures. Once a robot has exploded it will not
recover. There are no Byzantine robots in the group (we hope).


\subsection*{Broadcast, failure detection and lost messages} % (fold)
\label{subsec:lost_messages}

Your design may require the use of a failure detector. This may cause problems
in broadcast algorithms if correct robots are falsely suspected due to setting
the timeout of \emph{P} too low.

In order to select a time bound for \emph{P}, determine how long is a
reasonable time for a perfectly delivered message to be sent (heartbeat
request) and returned (heartbeat response) such that some number of message
losses is allowed.

For this problem you may assume a synchronous network: an upper bound of 2.5
seconds on a single message's delivery, so 5 seconds round-trip time without
losses. However, 1 in 10 messages are lost by the \emph{fll} connecting the
robots. Reason probabilistically about these message losses, select a number
of messages-in-a-row you are prepared to say will never be lost (say, 3
successive omission failures) by a correct process, and take that as the
maximum retransmissions required by your perfect link abstraction to achieve
delivery. Include this reasoning in your report. You may find the performance
sections of \cite{cachin2011}'s algorithm descriptions useful in determining
the time taken.

% If using the perfect failure detector, set its timeout so that 2 messages can
% be lost without failure being detected. The maximum round-trip time is 5
% seconds, but you must also consider the retransmission delay of \emph{sl}.

% subsection lost_messages (end)


\subsection*{Deliverables [4 points]} % (fold)
\label{sub:p2_deliverable}

Explain your solution, and parameters for your failure detector, if used, in
no more than 500 words.

Implement your solution in a module called \emph{crash\_robots}.

Your solution will be tested by killing robots during the performance. One way
to do this is to close their GUI window. Another might involve Erlang's kill
function. You should see ``stack terminated'' in the terminal confirming the
robot and its stack have stopped.

Again, the finishing position and dance sequence of robots will be checked,
but this time it will only be \emph{correct} robots (the ones which survived).
The \emph{fll} \verb!MAX_DELAY! will be 2500 ms, and \emph{sl}
\verb!RESEND_PERIOD! 3000 ms.


% subsection deliverable (end)

% section problem_2_failing_nodes (end)












\section*{Problem 3: Failing nodes in partial synchrony} % (fold)
\label{sec:problem_3_partial_synchrony}

In the previous scenario we assumed an upper bound on the network latency, and
determined a reasonable limit for transmission time from \emph{sl} to a
correct process. While this allowed us to use the fail-stop abstraction, it is
reliant on having a connection to the other robots which provides these
quality-of-service guarantees, which can be expensive and is not always
available.

We will now remove the upper-bound on transmission: messages can be delayed
arbitrarily. In the evaluation of your work we will think of a big number,
change the \emph{fll} maximum delay, and see if your algorithm works.

You should now decide whether to adopt the fail-noisy or fail-silent failure
abstraction, and consider the use of quorums and $\Diamond P$ or $\Omega$ to
maintain the robot invariants.

\subsection*{Number of expected failures} % (fold)
\label{sub:number_of_expected_failures}

Fortunately, the Shamshung engineers have now provided you with a predictive
model for the robot failures. They tell you that the number of failed robots
for a 100 step dance obeys a normal distribution with mean $1.5$ and variance
$0.4$\footnote{\url{http://www.wolframalpha.com/input/?i=normal+distribution+mean+1.5+variance+0.4}}.
Armed with this information you may reason about quorums and fail-silent
algorithms.

% subsection number_of_expected_failures (end)


\subsection*{Difficulty of quorums vs $\Diamond P$ or $\Omega$} % (fold)
\label{sub:difficulty_of_quorums_vs_diamond_p_}

Quorums provide a simpler solution, but limit the number of failures your
group of robots will survive. $\Diamond P$ and $\Omega$ will likely require
some amount of modifications to be made to the dancing robot application code.
If you can make this work with a correct algorithm it will be rewarded in the
marking.

% subsection difficulty_of_quorums_vs_diamond_p_ (end)


\subsection*{Deliverable [5 points]} % (fold)
\label{sub:p3_deliverable}

Explain your solutions in your report in no more than 700 words.
Discuss the problems for the application introduced by unbounded message
delays and the fail-noisy abstraction (recovering suspected nodes).
Include a discussion of the useful properties of any new
communication abstractions you implement and use.
State any assumptions and limitations of your solution, in particular those
related to the \emph{entertainment} and \emph{halting} requirements.

You may determine that this is not possible with the
current requirements. If so, argue your case, and suggest changes to the
specification in addition to your solution.

Implement your solution in the module \emph{ps\_crash\_robots}.
Your solution will be tested in the same fashion as Problem 2, but with
an unspecified \emph{fll} maximum delay. The number of failures injected will
depend on your design. You should explain any limitation here. For example,
what happens if all but one of the robots fails, will the dance complete?

% subsection deliverable (end)

% section problem_3_partial_synchrony (end)










\section*{Extension: ``They've got no rhythm'' [5 points]} % (fold)
\label{sec:buffered_dance_routine}

While we have solved the issue of the robots dancing the same steps in the
same order, they perform dance moves on the delivery of messages; this isn't
ideal, since messages can be arbitrarily delayed, which will mean each robot
dancing with its own rhythm: they'll be doing the same thing, but something
just won't look right, and how will the marketing team know what kind of music
to play?

Think of a way to modify the application so that the robots execute dance
moves at regular intervals, while maintaining the invariants. Decide on an
interval, such as 2 steps per second. You might derive this from the
beats-per-minute of your favourite music. A key problem here is ensuring the
robots know what the next step is in advance of taking it, without violating
the \emph{spontaneity} requirement.

Document your design in no more than 2 pages. We will experiment with your
implementation \emph{timed\_robots} according to your written report. The
robots should still keep to the same sequence of dance moves, but now also
move at regular intervals -- this will be checked visually.

You may find it simpler to start with the assumption that no robots crash
(problem 1), and gradually add the failures of problem 2 and 3. State which
crash model your robots work for. The harder failure models are worth more
points.



% section buffered_dance_routine (end)






\section*{Extension: optimising the point-to-point links [2 points]} % (fold)
\label{sec:extension_optimising_the_link_channels}

The \emph{pl}, \emph{sl}, \emph{fll} messaging stack is theoretically correct,
but expensive in practice. Can you modify it so that fewer messages are sent,
without risking that messages are not delivered?

Can you think of a way to safely garbage-collect the stored and retransmitted
messages? How else might you optimise \emph{pl}, \emph{sl}, \emph{fll}? Refer to the
performance clauses of sections 2.4.3-4 of \cite{cachin2011} for ideas.
Consider the consequences of node failures and the various failure models
with the algorithms and your changes.

Deliver the files as usual, and also a discussion of what you have done and
why you believe it is correct in the report. Check your robots still work
with the changes you have made (as we will only test and mark it with your
optimisations in place).

% section extension_optimising_the_link_channels (end)






\section*{Extension: same-room dancing robots [2 points]} % (fold)
\label{sec:extension_same_room_dancing_robots}

Consider again the close-proximity robots in this video:
\url{https://www.youtube.com/watch?v=4t1NWH6G1f0}. During performances,
members of the public commented that some of the robots seemed to be dancing
out-of-time with the others.

Describe a mechanism for keeping the group of
robots (while dancing in the same room, assuming a private and uncongested
network interconnection) in time with each other, without recourse to a
hypothetical global clock. Discuss which of the previous designs would be most
appropriate in this setting, and any changes you might make to it.
You are not expected to implement this solution.

% section extension_same_room_dancing_robots (end)


















\begin{thebibliography}{9}

\bibitem{cachin2011}
  C. Cachin, R. Guerraoui, and L. Rodrigues,
  \emph{Introduction to Reliable and Secure Distributed Programming}.
  Springer-Verlag, Berlin
  2nd Edition,
  2011.

\end{thebibliography}

\end{document}


