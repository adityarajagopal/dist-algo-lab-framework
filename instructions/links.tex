% TO BE DONE 19th April
% ALSO GIVE LAB ASSIGNMENT ON 19th APRIL
% 1st deadline: 26th
% final: 3rd May

% How is it graded
% How did they split the work

%
%  untitled
%
%  Created by Nicholas Trevor Rutherford on 2013-03-19.
%  Copyright (c) 2013 __MyCompanyName__. All rights reserved.
%
\documentclass[a4paper]{article}

% Use utf-8 encoding for foreign characters
\usepackage[utf8]{inputenc}

% Setup for fullpage use
% \usepackage{fullpage}

% Uncomment some of the following if you use the features
%
% Running Headers and footers
%\usepackage{fancyhdr}

% Multipart figures
%\usepackage{subfigure}

% More symbols
% \usepackage{amsmath}
% \usepackage{amssymb}
%\usepackage{latexsym}
\newcommand{\union}{\cup}

% Surround parts of graphics with box
% \usepackage{boxedminipage}

% Package for including code in the document
\usepackage{listings}

\usepackage[ruled,vlined]{algorithm2e}

\usepackage{hyperref}


% This is now the recommended way for checking for PDFLaTeX:
\usepackage{ifpdf}

%\newif\ifpdf
%\ifx\pdfoutput\undefined
%\pdffalse % we are not running PDFLaTeX
%\else
%\pdfoutput=1 % we are running PDFLaTeX
%\pdftrue
%\fi

\ifpdf
\usepackage[pdftex]{graphicx}
\else
\usepackage{graphicx}
\fi
\title{LSINF2345 Communication links abstraction lab}
\author{Nicholas Rutherford}

\date{19 March 2013}


\begin{document}
\lstset{language=erlang}

\ifpdf
\DeclareGraphicsExtensions{.pdf, .jpg, .tif}
\else
\DeclareGraphicsExtensions{.eps, .jpg}
\fi

\maketitle


\section{Introduction}

In this lab you will explore the software framework provided
for your coursework assignments. It includes a component stack model, and the
fair-loss and stubborn link modules as a starting point and templates to work
from.

Attempt the following exercises in the same pairs you will use for your graded
coursework. As for the coursework, it's ok to talk to the other students about
how to solve the problems, but don't copy their code.

\section{Framework overview} % (fold)
\label{sec:framework_overview}

This section briefly explains the role of key framework files.

In order to find documentation for Erlang functions, try Google, the reference
manual: \\
\url{http://www.erlang.org/doc/apps/stdlib/index.html}, and possibly
\url{http://erldocs.com}.

\subsection{Components} % (fold)
\label{sub:components}

The following components implement the two basic point-to-point communication
abstractions provided by the book. You should use these files as templates
for your own components implementing the higher layer abstractions.

\subsubsection{fll.erl Fair-loss link} % (fold)
\label{ssub:fll_erl}

Fair-loss links provide best-effort point-to-point communication between
discrete processes (nodes). Recall that while this link is unreliable, it is
used by the higher level components, which provide stronger abstractions such
as reliable delivery.

Hopefully you will find that the component's implementation looks very
similar to the algorithm presented in the course book.

%TODO cite book 2nd edition, not 1st, module names changed

The details of inter-node communication have been abstracted away in the stack
module. You should assume best-effort (unreliable) communication, as described
for the fll component. That is, a message may fail to arrive, and no ordering
guarantees or time bounds are provided, but some messages will arrive (it's
unreliable, but not completely useless).\footnote{For the curious, Erlang
primitives build on TCP and tend to give stronger guarantees, so for testing
purposes we have injected additional problems by randomly dropping and
delaying sent transmissions.}

If you'd like to use \lstinline!io:format! yourself to make your program more
chatty, refer to the documentation at
\url{http://erlang.org/doc/man/io.html#format-1}\\
or the side note at \url{http://learnyousomeerlang.com/syntax-in-functions}
to find out about the string substitution characters \lstinline!'~w~n'!.

% subsubsection fll_erl (end)

\subsubsection{sl.erl Stubborn link} % (fold)
\label{ssub:sl_erl_stubborn_link}

Stubborn links build on fair-loss links by guaranteeing that a sent message
will eventually arrive at its destination – in fact it will arrive many times.

Here you can see how stack:trigger is used to create the stack events
prescribed by the book's algorithms. Here it's used to request that fll sends
a message on its behalf. These are asynchronously broadcast to the other components,
which will act on them or ignore them according to (your implementation of)
their algorithm.

Note how a timer is used to periodically trigger the component's timeout event
and re-send the messages. Also see the sets documentation for methods that
might be useful in your own code. You might prefer to use lists, or to read
the overview of available data structures at
\url{http://learnyousomeerlang.com/a-short-visit-to-common-data-structures}.

In case you're not familiar with higher-order functions, \lstinline!sets:fold!
is used to execute the given function once for each set element. We don't use
the returned accumulator value as here we were only interested in fold as an
iterator, not an aggregator. The lists library also provides map and some
other functions, as you can find in the documentation.

% subsubsection sl_erl_stubborn_link (end)

% subsection components (end)



\subsection{Internals} % (fold)
\label{sub:internals}

In order to understand the implementation of the provided abstraction you
may find the following useful:
\url{http://www.erlang.org/doc/man/gen_server.html}\\
\url{http://learnyousomeerlang.com/clients-and-servers}

\subsubsection{stack.erl} % (fold)
\label{sub:stack_erl}

The stack module is responsible for loading components, propagating events
within the stack, and ensuring the stack is the unit-of-failure: when one
component crashes the entire stack stops.

% subsection stack_erl (end)

\subsubsection{component.erl} % (fold)
\label{sub:component_erl}

The component module is an abstraction to hide some irrelevant implementation
details from you regarding their startup and the propagation and receipt of
stack events.

% subsection component_erl (end)

% subsection internals (end)



\section{Lab activity} % (fold)
\label{sec:lab_activity}

The following exercises will have you implement your first components with the
framework.

\subsection{Implement perfect links} % (fold)
\label{sub:implement_pl}

In a new file, \verb!src/pl.erl!, implement the ``Perfect Link'' abstraction
with Algorithm~\ref{algo:pl}. Use the sl component as a template.

%TODO: insert the algorithm here

% \begin{verbatim}
% Algorithm 2.2: Eliminate Duplicates
% Implements: PerfectPointToPointLinks, instance pl.
% Uses: StubbornPointToPointLinks, instance sl.
%
% upon event < pl, Init > do
%   delivered := emptyset;
%
% upon event <pl, Send | q, m > do
%   trigger < sl, Send | q, m >;
%
% upon event <sl, Deliver | p, m > do
%   if m in delivered then
%     delivered := delivered union {m};
%     trigger < pl, Deliver | p, m>;
% \end{verbatim}

\begin{algorithm}[htbp]
  \SetKw{KwImplements}{Implements: }{}{}
  \SetKw{KwUses}{Uses: }{}{}
  \SetKw{KwTrigger}{trigger}
  \SetKwProg{KwEvent}{upon event}{ do}{}

  \SetAlgoNoLine % \SetAlgoLined
  \KwImplements{PerfectPointToPointLinks, instance pl}.\\
  \KwUses{StubbornPointToPointLinks, instance sl}.\\
  \BlankLine

  \KwEvent{$<$ pl, Init $>$}{
    $delivered := emptyset$\;
  }
  \BlankLine

  \KwEvent{$< pl, Send | q, m >$}{
    \KwTrigger{$< sl, Send | q, m >$}\;
  }
  \BlankLine

  \KwEvent{$< sl, Deliver | p, m >$}{
    \uIf{m in delivered}{
      $delivered := delivered \union \{m\}$\;
      \KwTrigger{$< pl, Deliver | p, m>$}\;
    }
  }
  \BlankLine
 \caption{Eliminate Duplicates, by Cachin et al.}
 \label{algo:pl}
\end{algorithm}


% subsection implement_pl (end)

\subsection{Optimising the perfect-link abstraction} % (fold)
\label{sub:optimising_pl}

The pl, sl, fll messaging stack is theoretically correct, but expensive to
operate as-is. In the following you'll optimise your stack to send fewer
messages without jeopardising the resulting abstractions.

\subsubsection{Weakening stubborn delivery by halting retransmission} % (fold)
\label{ssub:weakening_stubborn_delivery_by_halting_retransmission}

Does the ``stubborn delivery'' property make sense in practice? Can you think
of a time when it's safe to stop retransmitting the same message? How would
you implement it? Can you do it without modifying the sl.erl file?

% subsubsection weakening_stubborn_delivery_by_halting_retransmission (end)

\subsubsection{Garbage collecting terminated algorithm retransmissions} % (fold)
\label{ssub:halting_retransmission_on_algorithm_termination}

Having found a way to halt the retransmission of individual messages, can you
think of a way to modify this to work on sets of messages? Once an algorithm
using the links has terminated execution there is no need to continue
retransmitting its messages. For now don't worry about algorithm termination,
simply demonstrate that you can group messages by an integer identifier (or
similar) and have them all stop being retransmitted. That is to say, you
should modify ``If a correct process p sends a message m once to a correct
process q, then q delivers m an infinite number of times'' to ``If a correct
process p sends a message m once to a correct process q as part of an
algorithm a, then q delivers m one or more times before a terminates, and
infinitely often if a does not terminate''.

% subsubsection halting_retransmission_on_algorithm_termination (end)

\subsubsection{Other ideas} % (fold)
\label{ssub:other_ideas}

How else might you optimise pl, sl, fll? Check the performance clauses of
sections 2.4.3-4 in the course book, about sl and pl, for ideas.

Consider the consequences of node failures and the various failure models
with the algorithms and your changes.

% subsubsection other_ideas (end)



% subsection optimising_pl (end)

% section lab_activity (end)

% section framework_overview (end)


\bibliographystyle{plain}
\bibliography{}



\end{document}






