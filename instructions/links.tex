% TO BE DONE 19th April
% ALSO GIVE LAB ASSIGNMENT ON 19th APRIL
% 1st deadline: 26th
% final: 3rd May

% How is it graded
% How did they split the work

\documentclass[a4paper]{article}

% Use utf-8 encoding for foreign characters
\usepackage[utf8]{inputenc}

% Setup for fullpage use
\usepackage{a4wide}

% Uncomment some of the following if you use the features
%
% Running Headers and footers
%\usepackage{fancyhdr}

% Multipart figures
%\usepackage{subfigure}

% More symbols
% \usepackage{amsmath}
% \usepackage{amssymb}
%\usepackage{latexsym}
\newcommand{\union}{\cup}

% Surround parts of graphics with box
% \usepackage{boxedminipage}

\usepackage[ruled,vlined]{algorithm2e}

\usepackage{hyperref}


% This is now the recommended way for checking for PDFLaTeX:
\usepackage{ifpdf}

%\newif\ifpdf
%\ifx\pdfoutput\undefined
%\pdffalse % we are not running PDFLaTeX
%\else
%\pdfoutput=1 % we are running PDFLaTeX
%\pdftrue
%\fi

\ifpdf
\usepackage[pdftex]{graphicx}
\else
\usepackage{graphicx}
\fi
\title{LSINF2345 Lab: Implementing the basic abstractions}
\author{Nicholas Rutherford}

\date{April 2013}


\begin{document}

\ifpdf
\DeclareGraphicsExtensions{.pdf, .jpg, .tif}
\else
\DeclareGraphicsExtensions{.eps, .jpg}
\fi

\maketitle


\section{Introduction}

This lab introduces the software framework provided to help
implement the algorithms described in the course \cite{cachin2011}.
It will provide a foundation for implementing your project.

You have been provided with an Erlang framework including a component model,
node configuration, and the fair-loss and stubborn link components. Using the
\emph{sl} module as a template, you will implement your own components: a
perfect-link, failure detector, and broadcast component.

\subsection{Working in groups} % (fold)
\label{sub:how_to_work}

Where possible, work in the same pairs as for the project. Components built in
this lab should be useful in building your project solution.

% subsection how_to_work (end)

\subsection{Installing Erlang} % (fold)
\label{sub:installing_erlang}

We suggest installing Erlang/OTP R16B or newer. Mac users should install
a 32-bit build so that wxWidgets (gui) works. Source and binaries can be found at
\url{https://www.erlang-solutions.com/downloads/download-erlang-otp} and
\url{http://www.erlang.org/download.html}.

% subsection installing_erlang (end)

\subsection{Where to find help} % (fold)
\label{sub:where_to_find_help}

Discuss problems with the assistant and the other students. However, don't
copy their code. Erlang documentation can be found with Google, or at the
reference
manual \cite{manual_stdlib}\cite{manual_expressions} or
\url{http://erldocs.com}. Algorithms can be found in the course slides and
textbook \cite{cachin2011}.

% subsection where_to_find_help (end)



\section{Framework introduction} % (fold)
\label{sec:framework_overview}

This section briefly explains the role of key framework files. All files can
be found in the src directory. Concentrate on sl.erl, which provides
an example of implementing an algorithm from \cite{cachin2011}'s programming
model. The other files may be treated as black boxes for the lab and
project.

We provide two basic point-to-point communication abstractions,
\emph{fll} and \emph{sl}. You should use \emph{sl} as a template for your
higher-layer components.

\subsection{Programming with the framework} % (fold)
\label{sub:api}

In order to implement your own components (modules):

\begin{itemize}

\item Create a new module sharing the name of your component, e.g. module(pl)
in file `pl.erl'

\item Copy the \verb!-export! directive, \verb!start_link! and \verb!stop!
functions, and base-case of \verb!upon_event! which handles events not
matching any other

\item Copy the state record definition, and change it to suit your component
(or make it empty, \verb!-record(state, {}).!)

\item Implement the uses function, and include the components your module
depends on (as listed in its algorithm)

\item Reproduce the code from sl if you have initialisation or timeout events

\item Implement the messaging events and triggers required by the algorithm

\end{itemize}

Some key functions you will use:

\begin{description}
  \item[stack:trigger] to send events to other components on the stack
  \item[component:start\_timer(N)] request a timeout callback event every N milliseconds
  \item[stack:nodes] to return $\Pi$, all nodes in the system, in sorted order

  \item[stack:start\_link] launch the stack in a running Erlang process

  \item[stack:add\_component] add a component (and its dependencies) to the
  stack. Used components auto-loaded if missing

  \item[stack:connect] connect to a remote Erlang process, globally updating $\Pi$.
      The value connected to should be the result of running node() on that process, or the name the remote server was launched with
\end{description}

% subsection api (end)

\subsection{Compiling} % (fold)
\label{sub:compiling_and_running_the_program}

From the project directory, build your code with \verb!./rebar clean compile!.
To also run tests, \verb!./rebar clean compile eunit!.

You will need to recompile each time you make changes, and for simplicity we
suggest shutting down and restarting all processes each time you recompile.
To save typing you can paste multiple lines of commands into the erl shell at
a time, also multiple commands on one line, just remember to follow each
statement with a ``.''.


\subsection{Running your stack} % (fold)
\label{sub:running_your_stack}

% subsection running_your_stack (end)

To test with Erlang erl terminal shells, for each process (node):

\begin{itemize}

  \item Open a new terminal and cd to the project directory

  \item \verb!erl -sname 'n1@localhost'! -- launch erlang processes (for n1..5)

  \item \verb!cd(ebin).! -- Move the Erlang process pwd to your binaries directory

  \item \verb!stack:start_link().! -- Spawn the stack

  \item \verb!stack:connect('n1@localhost').! -- connect to the cluster; check
  with \verb!stack:nodes().!

  \item \verb!stack:add_component(pl).! -- launch the top-level stack component

\end{itemize}

Now pick a terminal, remembering that each is a different process in the
distributed system, and start manually issuing events. For example:

\begin{itemize}
  \item Send a message with \emph{sl}: \verb!stack:trigger({sl, send, lists:nth(1, stack:nodes()), knock_knock}).!
  \item Broadcast with \emph{beb}: \verb!stack:trigger({beb, send, hejhej}).!
\end{itemize}

Add some \verb!io:format! statements in component delivery events to see
whether your messages are arriving at each of your nodes. Remember to
recompile.

% subsection compiling_and_running_the_program (end)

\subsection{fll.erl Fair-loss link} % (fold)
\label{ssub:fll_erl}

Fair-loss links provide best-effort point-to-point communication between
discrete processes (nodes). Recall that while this link is unreliable, it is
used by the higher level components, which provide stronger abstractions such
as reliable delivery.

The details of inter-node communication are handled by the stack module. You
should assume best-effort (unreliable) communication, as described for the
\emph{fll} component. That is, a message may fail to arrive, and no ordering
guarantees or time bounds are provided, but some messages will arrive --
unreliable, but not useless\footnote{Erlang's primitives build on TCP and tend
to give stronger guarantees, so for testing purposes we have injected
additional problems by randomly dropping and delaying sent and received
transmissions.}.

% subsubsection fll_erl (end)

\subsection{sl.erl Stubborn link} % (fold)
\label{ssub:sl_erl_stubborn_link}

Stubborn links build on fair-loss links by guaranteeing that a sent message
will eventually arrive at its destination; in fact it will arrive many times.

Note how stack:trigger is used to create the events prescribed by the book's
algorithms. Here it's used to request that \emph{fll} sends a message on its
behalf. These are asynchronously broadcast to the other components, which will
act on them or ignore them according to (your implementation of) their
algorithm.

Note how a timer is used to periodically trigger the component's timeout event
and re-send the messages. Also see the sets documentation for methods that
might be useful in your own code. You might prefer lists, or some other
structure\footnote{\url{http://learnyousomeerlang.com/a-short-visit-to-common-data-structures}}.

In case you're not familiar with higher-order functions, sets:fold is used to
execute the given function once for each set element (the result is
discarded). The different data structures provide their own functions.

As a final point, examine the tagging of sent messages and filtering of
received messages. This is important, as without the means to identify which
messages were sent by \emph{sl}, it would also deliver messages not sent by \emph{sl},
violating the no-creation property. Other solutions might involve
point-to-point messaging rather than broadcasting of events within the stack.
Here we retain the notion that all events are seen by all components, and
rely on them structuring and filtering their messages accordingly.

% \subsection{Garbage collecting terminated algorithm retransmissions}
% \label{ssub:halting_retransmission_on_algorithm_termination}
%
% Having found a way to halt the retransmission of individual messages, can you
% think of a way to modify this to work on sets of messages? Once an algorithm
% using the links has terminated execution there is no need to continue
% retransmitting its messages. For now don't worry about algorithm termination,
% simply demonstrate that you can group messages by an integer identifier (or
% similar) and have them all stop being retransmitted. That is to say, you
% should modify ``If a correct process p sends a message m once to a correct
% process q, then q delivers m an infinite number of times'' to ``If a correct
% process p sends a message m once to a correct process q as part of an
% algorithm a, then q delivers m one or more times before a terminates, and
% infinitely often if a does not terminate''.



% subsubsection sl_erl_stubborn_link (end)

\subsection{Internals} % (fold)
\label{sub:internals}

The following shouldn't be needed to complete the lab, but may be of interest.
In order to understand the inner workings of the implementation (not required)
you may find the following useful: \cite{man_gen_server},
\cite{lyse_client_server}. The stack module is responsible for loading
components, propagating events within the stack, and ensuring the stack is the
unit-of-failure: when one component crashes the entire stack stops.

The component module is an abstraction to hide the implementation details of
component startup and the receiving of events.

% subsection internals (end)

% section framework_overview (end)




\section{Implement perfect links} % (fold)
\label{sub:implement_pl}

In a new file, \verb!src/pl.erl!, implement the ``Perfect Link'' abstraction
with Algorithm~\ref{algo:pl}, ``Eliminate Duplicates''. Use the \emph{sl}
component as a template.

% \begin{verbatim}
% Algorithm 2.2: Eliminate Duplicates
% Implements: PerfectPointToPointLinks, instance pl.
% Uses: StubbornPointToPointLinks, instance sl.
%
% upon event < pl, Init > do
%   delivered := emptyset;
%
% upon event <pl, Send | q, m > do
%   trigger < sl, Send | q, m >;
%
% upon event <sl, Deliver | p, m > do
%   if m in delivered then
%     delivered := delivered union {m};
%     trigger < pl, Deliver | p, m>;
% \end{verbatim}

\begin{algorithm}[htbp]
  \SetKw{KwImplements}{Implements: }{}{}
  \SetKw{KwUses}{Uses: }{}{}
  \SetKw{KwTrigger}{trigger}
  \SetKwProg{KwEvent}{upon event}{ do}{}

  \SetAlgoNoLine % \SetAlgoLined
  \KwImplements{PerfectPointToPointLinks, instance pl}.\\
  \KwUses{StubbornPointToPointLinks, instance sl}.\\
  \BlankLine

  \KwEvent{$<$ pl, init $>$}{
    $delivered := \emptyset$\;
  }
  \BlankLine

  \KwEvent{$< pl, send, q, m >$}{
    \KwTrigger{$< sl, send, q, m >$}\;
  }
  \BlankLine

  \KwEvent{$< sl, deliver, p, m >$}{
    \uIf{m in delivered}{
      $delivered := delivered \union \{m\}$\;
      \KwTrigger{$< pl, deliver, p, m>$}\;
    }
  }
  \BlankLine
 \caption{Eliminate Duplicates, by Cachin et al.}
 \label{algo:pl}
\end{algorithm}


\subsection{Maintain no-creation with shared components} % (fold)
\label{ssub:no_creation_with_multiple_stack_algorithms}

Modify the algorithm so that \emph{pl} only delivers its own messages, maintaining
the no-creation property. It should ignore \emph{sl} deliver events for messages not
sent by \emph{pl} (where \emph{pl} may be itself, or the \emph{pl} component
on another node). \emph{Hint: look for something simple, you don't need to
modify the framework}.


\subsection{Duplicate messages} % (fold)
\label{ssub:duplicate_messages}

Sometimes we want to send the same message twice. Take, for example, the
perfect failure detector's \emph{heartbeat\_reply} message. The message
content is identical each time a heartbeat reply is sent, but the messages are
independent. Each time a heartbeat is sent by \emph{pl} it should be delivered
exactly once. If the same message content is sent twice it should be
delivered exactly twice, otherwise the failure detector would suspect all nodes at
the second heartbeat. Messages should be distinguished by the act of sending
them, not by their content.

Modify the algorithm so that if two requests arrive for the same message
the message is sent exactly twice. That is, each received send request
produces a unique message even if the contents of the sent messages are equal.
You may find
\verb!make_ref()!\footnote{\url{http://erlang.org/doc/man/erlang.html\#make_ref-0}}
useful for generating pseudo-random ids, but this may not be the best approach
if introducing FIFO ordering or garbage collection.

% subsubsection duplicate_messages (end)


\subsection{Best-effort broadcast} % (fold)
\label{sub:best_effort_broadcast}

Having implemented \emph{pl}, implement the best-effort broadcast component,
\emph{beb}. Test it by sending messages from the erl terminal to several
nodes.

% subsection best_effort_broadcast (end)

% subsubsection no_creation_with_multiple_stack_algorithms (end)


% subsection implement_pl (end)


\section{Extra: failure detector, leader election, reliable broadcast} % (fold)
\label{sub:further_work}

Having demonstrated your perfect links and best-effort broadcast to the
assistant, implement the perfect failure detector.

In a new file, \verb!src/p.erl!, implement the ``Perfect Failure Detector''
abstraction with Algorithm~\ref{algo:p}, ``Exclude on Timeout''. Test the
failure detector by connecting two or more nodes running \emph{P} and
observing the behaviour of the other nodes when you shut one down by
\verb!stack:stop()! or somehow killing the Erlang process.

Use your failure detector to implement lazy reliable broadcast and perfect
leader election.


\begin{algorithm}[htbp]
  \SetKw{KwImplements}{Implements: }{}{}
  \SetKw{KwUses}{Uses: }{}{}
  \SetKw{KwTrigger}{trigger}
  \SetKwProg{KwEvent}{upon event}{ do}{}

  \SetAlgoNoLine % \SetAlgoLined
  \KwImplements{PerfectFailureDetector, instance p}.\\
  \KwUses{PerfectPointToPointLinks, instance pl}.\\
  \BlankLine

  \KwEvent{$<$ p, Init $>$}{
    $alive := \Pi$\;
    $detected := \emptyset$\;
    $starttimer(\Delta)$;
  }
  \BlankLine

  \KwEvent{$<Timeout>$}{
    \ForEach{$proc\in\Pi$}{
      \uIf{$(proc \notin alive) \land (proc \notin detected) $}{
        $detected := detected \union \{proc\}$\;
        \KwTrigger{$< p, crash, proc>$}\;
      }
      \KwTrigger{$< pl, send, proc, heartbeat\_request >$}\;
    }
    $alive := \emptyset$\;
    $starttimer(\Delta)$\;
  }
  \BlankLine

  \KwEvent{$<pl, deliver, q, heartbeat\_request>$}{
    \KwTrigger{$<pl, send, q, heartbeat\_reply>$};
  }
  \BlankLine

  \KwEvent{$<pl, deliver, proc, heartbeat\_reply>$}{
    $alive := alive \union \{proc\}$;
  }

 \caption{Exclude on Timeout, by Cachin et al.}
 \label{algo:p}
\end{algorithm}



% section lab_activity (end)





\section{Some Erlang pointers} % (fold)
\label{sec:some_erlang_pointers}

\subsection{Uppercase and lowercase terms} % (fold)
\label{sub:uppercase_and_lowercase_terms}

Lowercase terms are atoms, uppercase for variables. Function and module names
are atoms. Refer to \cite{manual_expressions}.

% subsection uppercase_and_lowercase_terms (end)

\subsection{Commas, semicolons, full-stops, conditional logic} % (fold)
\label{sub:commas_semicolons_full_stops}

As new Erlang programmers you will likely fall over the syntax rules for `,'
`;' `.' while writing your program. The compiler error messages can help to
find these problems. Also, refer to the manual \cite{manual_functions}, to
\cite{lyse_function_syntax}, and look at existing code.

Be sure to check how if and case statements work before using them. They
return values, like Ruby but unlike C. Case statements seem the most general,
and it is sufficient to learn to use those. Erlang case statements have little
in common with their C or Java counterpart, and do not execute multiple
cases by ``falling-through''.

% subsection commas_semicolons_full_stops (end)

\subsection{Print to terminal} % (fold)
\label{sub:debug_printing}

To make your program more chatty, perhaps for execution tracing and debugging,
you will find io:format useful. Refer to
the documentation\footnote{\url{http://erlang.org/doc/man/io.html\#format-1}}
or the side note at \cite{lyse_function_syntax} to
find out about the string substitution characters \verb!~w~n!.

% subsection debug_printing (end)

\subsection{Using the debugger} % (fold)
\label{sub:using_the_debugger}

To use the graphical debugger \cite{manual_debugger}, ensure
\verb!{erl_opts, [debug_info]}.! is present in the rebar.config file.
Having compiled with debug flags, in the erl terminal run \verb!debugger:start().!
then in \emph{Module$\rightarrow$Interpret} navigate to the src directory and select the
module(s) you want to observe.

% subsection using_the_debugger (end)

\subsection{Unit testing} % (fold)
\label{sub:unit_testing}

Automated testing is not required for the project, but you may find it useful.
If you would like to unit test your components you should look at the test
framework
EUnit\footnote{\url{http://www.erlang.org/doc/apps/eunit/chapter.html}}, and
the mocking library meck\footnote{\url{https://github.com/eproxus/meck}}. For
advice on where to write the tests and how to structure them also see
\cite{mochimedia_test} and \cite{so_eunit_q}.

You would then call your component's \verb!upon_event! function and make
assertions about the calls it makes to the (mocked) stack:trigger function,
and any other things of interest. You can run the tests with
\verb!./rebar eunit!

% subsection unit_testing (end)

% section some_erlang_pointers (end)





\begin{thebibliography}{9}

\bibitem{cachin2011}
  C. Cachin, R. Guerraoui, and L. Rodrigues,
  \emph{Introduction to Reliable and Secure Distributed Programming}.
  Springer-Verlag, Berlin
  2nd Edition,
  2011.

% \bibitem{cesarini_thompson_2009}
% Francesco Cesarini and Simon Thompson
% \emph{Erlang Programming, A Concurrent Approach to Software Development}
% O'Reilly Media
% 2009.

\bibitem{manual_stdlib}
  \emph{Erlang STDLIB Reference Manual}\
  \url{http://www.erlang.org/doc/apps/stdlib/index.html}

\bibitem{manual_expressions}
  \emph{Erlang Reference Manual}
  User's Guide, Expressions\
  \url{http://erlang.org/doc/reference_manual/expressions.html}.

\bibitem{manual_functions}
  \emph{Erlang Reference Manual}
  User's Guide, Function Declaration Syntax,\
  \url{http://erlang.org/doc/reference_manual/functions.html}

\bibitem{manual_debugger}
  \emph{Erlang Reference Manual}
  Debugger
  User's Guide\
  \url{http://www.erlang.org/doc/apps/debugger/debugger_chapter.html}



\bibitem{lyse_function_syntax}
  \emph{Learn you some Erlang},
  Syntax in functions\
  \url{http://learnyousomeerlang.com/syntax-in-functions}

\bibitem{man_gen_server}
  \emph{Erlang Reference Manual}
  User's Guide, Generic Server Behaviour
  \url{http://www.erlang.org/doc/man/gen_server.html}

\bibitem{lyse_client_server}
  \emph{Learn you some Erlang},
  Clients and servers
  \url{http://learnyousomeerlang.com/clients-and-servers}

\bibitem{mochimedia_test}
  \emph{meck and eunit best practices},
  Mochilabs
  \url{http://labs.mochimedia.com/archive/2011/06/13/meck-eunit-best-practices/}

\bibitem{so_eunit_q}
  \emph{Best practices/conventions for writing Erlang unit tests using eunit},
  Stackoverflow question
  \url{http://stackoverflow.com/questions/6449681/best-practices-conventions-for-writing-erlang-unit-tests-using-eunit}

\end{thebibliography}

\end{document}






